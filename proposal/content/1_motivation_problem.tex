% !TeX root = ..\Proposal.tex

\section{Motivation, Problem and Research Questions}

\subsection{Motivation \& Problem}
\begin{itemize}
  \item style is an important aspect of text
  \item it dictates how the text is perceived and how well the message is conveyed and accepted by the reader
  \item manual style annotation is very difficult and time-consuming, which is why there currently exists no large-scale annotated dataset for style
  \item current style represantion methods use unsupervised learning which leads to style representations that are not interpretable
  \item this increases the difficulty of verifying the quality of the style representation and using it in downstream tasks
  \vspace{1em}
  \item LLMs are increasingly being used to explain various topics to a wide and diverse audience
  \item these explanations would benefit from being in a style that fits the audience
  \item this is possible with current style represantion methods, but the lack of interpretability makes it difficult to verify the quality of the style representation
\end{itemize}

\subsection{Research Questions}
\begin{enumerate}
  \item How well does the classification model for interpretable style representations presented in \citet{patelLearningInterpretableStyle2023} work on group specific data?
  \item What is the best way to generate group specific explanation from style representations?
  \begin{itemize}
    \item mention the group in the prompt (e.g. \enquote{Write the following explanation in the style of a teenager})
    \item use the dimensions of the interpretable style vector in the prompt
    \item use the classifier for interpretable style representations to create data to implement the ActAdd method presented in \citet{turnerActivationAdditionSteering2024}
  \end{itemize}
\end{enumerate}
