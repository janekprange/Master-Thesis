% !TeX root = ..\Proposal.tex

\section{Motivation, Problem, and Research Questions}

\subsection{Motivation \& Problem}
Information in textual form is an essential part of our daily lives, be it in educational settings, news, entertainment or social media. One important aspect of text is not only its content and what information is portrayed, but also the style in which it is written~\cite{wegmannSameAuthorJust2022}. Style dictates how the text is perceived and how well the message is conveyed and accepted by the reader. Additionally, it can be used to decide which author or group of people the text is written by.

The importance of style was recognized and researched in previous works~\cite{zhu-etal-2024-styleflow, ijcai2020p526,wegmannSameAuthorJust2022}. However, manual style annotation is very difficult and time-consuming, which is why there currently exists no large-scale annotated dataset for style. State-of-the-art style representation methods use unsupervised learning which produces style embeddings that are not interpretable. This increases the difficulty of verifying the quality of the style representation and using it in downstream tasks.

Style representations can be used to predict the author of the text by comparing the stylistic features between documents~\cite{alshomaryLatentSpaceInterpretation2024}. Additionally, it is possible use style to detect group membership~\cite{10.1007/978-3-642-29047-3_27}. % TODO: Example of forensic setting for authorship attribution? Other example for usefulness?

In addition to using style representations to recognize authors or group membership, they also play a potentially important part for the generation of text. Large language models are increasingly being used to explain various topics to a wide and diverse range of audiences. While it is of course important that the explanations are factually correct, it is also necessary to consider the text style the audience is accustomed to and the background knowledge it has and steer the generated text appropriately. For example, a technical explanation for a PhD student would probably not be helpful for a middle schooler and vice versa.

There are existing methods to steer the style of generated text in a specific direction or transfer the style~\cite{zhu-etal-2024-styleflow, ijcai2020p526,konenStyleVectorsSteering2024,subramaniExtractingLatentSteering2022,turnerActivationAdditionSteering2024}. All of them have in common that they use training data where the content only differs in the steering target to fine-tune an LLM. However, because state-of-the-art style representation methods produce non-interpretable style embeddings, it is difficult to verify the quality of the style embedding and therefore error-prone to use the embedding to generate new text.

State-of-the-art authorship attribution methods as well as generation steering use mainly style attributes for its task. There are however other aspects about the author other than the content itself that can be extracted and assist these methods. This includes information about the background knowledge or experience of the author, which are subsequently called knowledge attributes.


\subsection{Research Questions}
\begin{enumerate}
  \item How well does the model for the generation of interpretable style representations presented in \citet{patelLearningInterpretableStyle2023} work on group-specific data?
  \item Does the interpretable style representation method benefit from knowledge attributes in addition to style attributes?
  \item What is the best way to generate group-specific explanations from style representations? \newline
  In this thesis I will explore two possible ways to achieve this:
  \begin{itemize}
    % \item Mention the group in the prompt (e.g. \enquote{Write the following explanation in the style of a teenager}), % baseline
    \item use the dimensions/attributes of the interpretable style vector in the prompt or
    \item use the style representation model to create training data to implement the ActAdd method presented in \citet{turnerActivationAdditionSteering2024}. It is however not yet clear if there is enough time to implement this approach.
  \end{itemize}
\end{enumerate}
