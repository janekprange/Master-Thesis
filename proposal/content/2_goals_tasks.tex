% !TeX root = ..\Proposal.tex

\section{Thesis Goals and Tasks to Tackle Each Goal}
  \subsection{Produce style sentences from data and train a style represantion model}
  \begin{itemize}
    \item in large parts a reimplementation of the style represantion model presented in \citet{patelLearningInterpretableStyle2023} with the addition of knowledge attributes (i.e. \enquote{The author has experience in \ldots})
    \item text generation with Llama3.2-3B-instruct~\cite{dubeyLlama3Herd2024}
    \item data from stackexchange and reddit; high probability of being written by a specific group of people
    \item test the resulting model by predicting if texts are from different groups using only the interpretable style embeddings
    \item for the test, I will use the STEL framework\cite{wegmann-nguyen-2021-capture}
  \end{itemize}
  %  \subsection{Extend the approach by including prompts aimed at producing descriptions that take the background knowledge and experience of the author/group into account.}
   \subsection{Implement different methods to generate text in a specific style with the help of the interpretable style vectors.}
   \subsection{Test how well this method works to generate text in the style of specific groups of people.}


% \begin{itemize}
%   \item using the python libraries numpy~\cite{harris2020array}, pandas~\cite{reback2020pandas,mckinney-proc-scipy-2010}, pytorch~\cite{paszkePyTorchImperativeStyle2019}, transformers~\cite{wolfHuggingFacesTransformersStateart2020}, sentence-transformers~\cite{reimersSentenceBERTSentenceEmbeddings2019} and nltk~\cite{birdNaturalLanguageProcessing2009}
% \end{itemize}
