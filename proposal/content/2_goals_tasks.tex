% !TeX root = ..\Proposal.tex

\section{Thesis Goals and Tasks to Tackle Each Goal}
  \subsection{Extract style attributes from data and train a style represantion model}
  This thesis is inspired heavily by \citet{patelLearningInterpretableStyle2023}. In their work, \citeauthor{patelLearningInterpretableStyle2023} use large language models (in their case ChatGPT) to extract style attributes in an unsupervised fashion. For each input text, the LLM is prompted around 90 times to describe it with a focus on different linguistic features. These style descriptions are than transformed into sentences of the form \enquote{The author uses x} or \enquote{The author is x} by prompting the LLM again.

  By using the similarity between the sentence embeddings of the style sentences to find positive and negative examples for specific style attributes, multiple models can be trained which produce an agreement score for one style attribute each. These models are then used to annotate a larger set of data and use that to train a model that produces interpretable style vectors, where each dimension of the vector is a style sentence.
  % with the addition of knowledge attributes. % TODO: describe method
  
  While the original method was used to differentiate the style from different authors, in my thesis I will use texts from different groups of people as the basis for my experiments. Additionally, I will try to extract not only style attributes, but knowledge attributes as well. Knowledge Attributes do not describe the style of the text, but express for example if the author seems to be experienced or has lots of background knowledge on some topic.
  
  In my thesis, I use Llama3.2-3B-instruct %~\cite{dubeyLlama3Herd2024}
  to extract the style sentences. The raw data that is the base for the synthetic dataset is taken from stackexchange and reddit. The forums included in the dataset all have the theme where a specific group of people (i.e. people over 30, engineers, \ldots) are asked questions. Because of this, there is a high probability that the answers are written by these groups so that the style differences can be extracted.

  The style represantion model will be tested on unseen answers from the forums. The model will be tasked with predicting if answers are from the same group of people by just using the interpretable style vector. Additionally, the STEL framework~\cite{wegmann-nguyen-2021-capture} will be used to test the style represantions against a human baseline.

  %  \subsection{Extend the approach by including prompts aimed at producing descriptions that take the background knowledge and experience of the author/group into account.}

   \subsection{Implement different methods to generate text in a specific style with the help of the interpretable style vectors.}
  While it will be useful to have a model than can produce interpretable style vectors for arbitrary text, an important task that I will tackle in this thesis is to use the style vector to generate new text in a specific style. There are three different ways to reach this goal that I want to compare.
  \begin{enumerate}
    \item Mention the group in the prompt (e.g. \enquote{Write the following explanation in the style of a teenager}). This is the baseline for all further experiments as the style vectors are not used. The style of the text generation can not be changed at a granular level, but it is very easy to implement.
    \item Use the dimensions/attributes of the interpretable style vector in the prompt. For this implementation, the x most important style attributes for a group a included as part of the prompt to steer the output. This is fairly easy to implement and may give more control of the generation style than the first method.
    \item Use the style represantion model to create training data to implement the ActAdd method presented in \citet{turnerActivationAdditionSteering2024}. ActAdd uses the difference in one activation layor of the model to steer its output. This method has the advantage that it needs very few training data and works even with one positive and one negative example. However, it was only used to steer the ouput on one attribute (e.g. \enquote{happy} vs. \enquote{sad}), not multiple as would be needed in the case of this thesis. This is by far the most complex method, which is why it is not yet certain that it will be possible to fully implement it as part of the thesis. It is however the most promising method as it would give fine-grained control over the amount that each style attribute plays into the generated text.
  \end{enumerate}

  %  \subsection{Test how well this method works to generate text in the style of specific groups of people.}


% \begin{itemize}
%   \item using the python libraries numpy~\cite{harris2020array}, pandas~\cite{reback2020pandas,mckinney-proc-scipy-2010}, pytorch~\cite{paszkePyTorchImperativeStyle2019}, transformers~\cite{wolfHuggingFacesTransformersStateart2020}, sentence-transformers~\cite{reimersSentenceBERTSentenceEmbeddings2019} and nltk~\cite{birdNaturalLanguageProcessing2009}
% \end{itemize}
