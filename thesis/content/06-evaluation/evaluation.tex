\chapter{Evaluation and Discussion}
\label{sec:evaluation}

\section{Evaluating the Importance of Knowledge Attributes for the Interpretable Attribute Vector}
\label{sec:evaluation:knowledgeAttributes}
\begin{itemize}
  \item the experiment is conducted for the most important \num{10}, \num{20} and \num{50} attributes per group
  \item the results of the experiment are shown in Table~\ref{table:knowledgeImportance}
  \item it is clear that the knowledge attributes are not equally important for all groups
  \item for some groups, such as Biologists, Historians and Politicians, the knowledge attributes are not important to distinguish them from other groups
  \item for most groups, there is at least some relevance to the knowledge attributes
  \item for the groups of Game Developers and Software Engineers, the knowledge attributes are very important as at least half of the \num{10} most important attributes are knowledge attributes
  \item this is especially significant since there are only \num{\minNumKnowledgePrompts} knowledge attributes in the \num{\styleVectorSize} dimensions of the attribute vector
  \item if the importance of knowledge and style attributes would be equally distributed, only less than \SI{10}{\percent} of the most important attributes would be expected to be knowledge attributes
  \item this shows that the extension of the state-of-the-art style vector proposed in this thesis does have a significant benefit for group membership detection and answers research question~\ref{rq:interpretableGroupDetect:knowledgeAttributes}
\end{itemize}
\begin{table}[ht]
  \caption[]{\begin{itemize}
      \item This table shows the results to the evaluation of the importance of knowledge attributes for group membership detection
      \item the attribute vector is created with the \ac{lisa} model for answers from the Stack Exchange dataset (see Section~\ref{sec:datasets:stackex})
      \item then, the dimensions that are the most important to differentiate each group from all others are selected and the number of knowledge attributes included in them counted
      \item the experiment shows that while knowledge attributes are not beneficial to differentiate the answers by some groups, overall it is very clear that they have significant importance for group membership detection
    \end{itemize}}
  \resultKnowledgeImportance{}%
  \label{table:knowledgeImportance}
\end{table}


\section{Testing Model Performance}
\label{sec:evaluation:models}
\begin{itemize}
  \item \textbf{\ac{sfam}} is evaluated by comparing its prediction if an attribute sentence matches an answer to the synthetic dataset (see Section~\ref{sec:experiments:setup:sentenceGeneration}) and seeing if the attribute sentence is actually used to describe the answer
  \item the results in Table~\ref{table:resultSFAM} show that \ac{sfam} has a solid ability to predict if an attribute sentence matches a text with an accuracy significantly higher than a random classifier
  \item even though \ac{sfam} follows the work of \citet{patelLearningInterpretableStyle2023}, the results can not be compared to their results as they did not test \ac{sfam} on a dataset of unseen texts but compared it to human judgments which was not possible in this thesis due to time constraints
  \item while the performance of \ac{sfam} is good, it could probably be improved by using a better model as the basis for fine-tuning and by reducing the noise in the test dataset that is the result of the data being extracted from internet forums
\end{itemize}

\begin{table}[ht]
  \caption[]{\begin{itemize}
      \item the performance of \ac{sfam} is evaluated by comparing its prediction if a attribute sentence matches a text with the fact if the sentence was used to describe the text in the synthetic dataset (see Section~\ref{sec:experiments:setup:sentenceGeneration})
      \item the experiment shows that \ac{sfam} performs significantly better than a random baseline
    \end{itemize}}
  \resultSfam{}%
  \label{table:resultSFAM}
\end{table}

\begin{itemize}
  \item \textbf{\ac{lisa}} is evaluated on attribute vectors that are created with \ac{sfam}
  \item the results in Table~\ref{table:resultLISA} show that \ac{lisa} is able to create attribute vector with only a small loss compared to \ac{sfam}
  \item however, the performance of the \ac{lisa} model by \citet{patelLearningInterpretableStyle2023} was a lot better with a \ac{mse} of \num{0.005}
  \item additionally, the average cosine similarity between the attribute vectors produced by \ac{sfam} and \ac{lisa} is not very high with \num{0.752}
  \item the performance is likely not as good because there is not enough training data
  \item a larger amount can not be used for this thesis for two reasons
        \begin{itemize}
          \item the Stack Exchange dataset that is used for training \ac{lisa} does not include many more answers that could be used for training
          \item the time constraints of this thesis do not allow a much longer training time
        \end{itemize}
\end{itemize}

\begin{table}[ht]
  \caption[]{\begin{itemize}
      \item \ac{lisa} is evaluated in comparison to attribute vectors created by \ac{sfam}
      \item accuracy and F1 are computed by using the attribute vectors by \ac{sfam} and \ac{lisa} to predict if each attribute matches the text; then, the predictions are compared to each other
    \end{itemize}}
  \resultLisa{}%
  \label{table:resultLISA}
\end{table}


The \textbf{embedding model} is tested on its ability to detect group membership on two tasks. The first evaluates whether an embedding is closer to another from the same group than to one from a different group (triplet-based evaluation), and the second assesses whether an embedding can be assigned to the correct group by comparing it to the median group embeddings. Both tasks use Euclidean and cosine distances, and are conducted on the Stack Exchange and AskX datasets to measure the generalization capabilities and robustness across domains of the model.

\begin{itemize}
  \item Table~\ref{table:embedder:triplet} shows that the model has robust ability to predict the correct group significantly better than a random classifier
  \item Table~\ref{table:embedder:medians} supports this finding. Even though the accuracy and F1 are significantly lower, the results are still good since it is not a binary classification anymore because there are \num{\numGroups} groups to choose from
  \item the main reason why the performance is not better is that some of the groups that are relativly similar overlap strongly in the embeddings space
  \item because of this, the accuracy of differentiating between for example politicians and computer scientists would be much higher than between software engineers and computer scientists

  \item the triplet-based evaluation on the foreign domain shown in Table~\ref{table:embedder:tripletForeignDomain} shows a robust performance for group membership detection
  \item the performance is a little worse than on the training domain as expected, but still good
  \item a similar conclusion can be drawn from the results of the evaluation based on median group embeddings on the foreign domain shown in Table~\ref{table:embedder:mediansForeignDomain}
  \item in this case, the metrics are even better than on the Stack Exchange dataset
  \item this is however likely the results of the fact that the AskX dataset has only \num{\numGroupsAskx} groups compared to the \num{\numGroups} groups of the Stack Exchange dataset
\end{itemize}

\begin{table}[ht]
  \caption{These tables show the performance of the embedding model to detect group membership based on group-specific answers.}
  \begin{subtable}[b]{0.49\linewidth}
    \subcaption[]{\begin{itemize}
        \item the performance of the embedding model in a triplet-based test
        \item here, the model is tested on wether an anchor embedding is closer to a positive embedding of the same group or to a negative embedding of a different group
        \item the model performance significantly better than a random baseline
      \end{itemize}}
    \resultEmbedder{}%
    \label{table:embedder:triplet}
  \end{subtable}
  \hfill
  \begin{subtable}[b]{0.49\linewidth}
    \subcaption[]{\begin{itemize}
        \item the performance of the model in assigning the correct group based on the median group embeddings
        \item the median embeddings are created from group-specific answers that are not used in the evaluation
        \item the experiment shows that the model is much more accurat than the random baseline
      \end{itemize}}
    \resultEmbedderToMedians{}%
    \label{table:embedder:medians}
  \end{subtable}
  \par\bigskip
  \begin{subtable}[b]{0.49\linewidth}
    \subcaption[]{\begin{itemize}
        \item the table shows the results of the triplet-based evaluation on the foreign domain of the AskX dataset (see Section~\ref{sec:datasets:askx})
        \item while the performance is slightly worse compared to the test shown in Table~\ref{table:embedder:triplet}, it still significantly better than the random baseline
      \end{itemize}}
    \resultEmbedderForeignDomain{}%
    \label{table:embedder:tripletForeignDomain}
  \end{subtable}
  \hfill
  \begin{subtable}[b]{0.49\linewidth}
    \subcaption[]{\begin{itemize}
        \item the performance of the model of assigning the correct group based on the median group embeddings on a foreign domain
        \item the median embeddings are created from group-specific answers that are not used in the evaluation
        \item the experiment shows that the model is significantly better than a random classifier and even better than the experiment shown in Table~\ref{table:embedder:medians}, although this is likely because the AskX dataset has only \num{\numGroupsAskx} groups in contrast to the \num{\numGroups} group of the Stack Exchange dataset
      \end{itemize}}
    \resultEmbedderToMediansForeignDomain{}%
    \label{table:embedder:mediansForeignDomain}
  \end{subtable}
\end{table}

\section{Testing Steering Performance}%
\label{sec:evaluation:steering}
\begin{itemize}
  \item both the prompt and activation steering methods are evaluated using the same questions from the steering dataset (see Section~\ref{sec:datasets:steering}) and compared to baseline (unsteered) explanations generated by a \acl{llm}
  \item the embeddings of the explanations are compared to median embeddings of the groups in the synthetic dataset (see Section~\ref{sec:experiments:setup:sentenceGeneration}), which act as a baseline of what style and background knowledge a specific group has
  \item Four metrics are used for evaluation: steering direction correctness, steering effect, optimal steering effect, and possible improvement (see Figure~\ref{fig:steeringMetrics})
\end{itemize}


\subsection{Activation Steering Hyperparameter Optimization}%
\label{sec:evaluation:steering:activationHPO}
% TODO: write about this in experiments
\begin{itemize}
  \item as described in Section~\ref{sec:approach:steering:activation:steering}, activation based steering has two hyperparameters
        \begin{itemize}
          \item the layer or layers the steering vectors are inserted into
          \item and the scalar \(\lambda\) that the steering vector is scaled with
        \end{itemize}
  \item while \citet{konenStyleVectorsSteering2024,bogdanEmergentEffectsScaling2025} have discovered that concepts with the complexity of attribute sentences are located in the layers in the middle of the model, that does not provide the best layer that should be used for steering
  \item to find the best values, a small hyperparameter optimization in the form of a grid search was conducted
  \item for \(\lambda\), the values \num{0.25} and \num{0.5} are chosen
  \item additionally, the steering capabilities with the layers \num{13}, \num{15}, and  \num{17} as well as the layer combinations with two layers at the same time  \num{13}, \num{14} and \num{16}, \num{17} and a combination with three layers \num{14}, \num{15}, \num{16}
  \item because of time constraints, a larger number of values can not be tested
  \item the metrics are the direction correctness and the strength of the steering effect as explained in Figure~\ref{fig:steeringMetrics}
\end{itemize}
\begin{itemize}
  \item Figure~\ref{fig:activationSteeringHPO} shows the result of the hyperparameter optimization
  \item for both metrics, the best layer is the combination of \num{14}, \num{15}, and \num{16}
  \item for the direction correctness, none of the single layers has a particularly good performance as can be seen in Figure~\ref{fig:activationSteeringHPO:directionCorrectness}
        \begin{itemize}
          \item this suggests that the experiment should be repeated with different layers to discover layers that hold more of the information that is necessary to steer explanations in the correct direction
        \end{itemize}
  \item for the steering effect, Figure~\ref{fig:activationSteeringHPO:steeringEffect} shows that the layer \num{13} has a very good performance
  \item the best performance is still the combination of three layers
  \item further experiments could show up to which point the performance is improved by steering on multiple layers at once
\end{itemize}

\begin{figure}[ht]
  \begin{subfigure}[c]{0.49\linewidth}
    \resizebox{\linewidth}{!}{
      \import{generated}{activation_steering_performance-direction_correctness.pgf}%
    }
    \subcaption{The direction correctness of the activation steering with different \(\lambda\) values and layers (higher is better).}%
    \label{fig:activationSteeringHPO:directionCorrectness}
  \end{subfigure}
  \hfill
  \begin{subfigure}[c]{0.49\linewidth}
    \resizebox{\linewidth}{!}{
      \import{generated}{activation_steering_performance-steering_effect.pgf}%
    }
    \subcaption{The steering effect of the activation steering with different \(\lambda\) values and layers (higher is better).}%
    \label{fig:activationSteeringHPO:steeringEffect}
  \end{subfigure}
  \caption{TODO:}%
  \label{fig:activationSteeringHPO}
\end{figure}


\subsection{Comparing different steering methods}

\begin{table}[b]
  % TODO: is the table highlighting correct? possible improvement has to be lowest highlighted
  \caption[]{\begin{itemize}
      \item This table shows the performance of different steering methods with the metrics shown in Figure~\ref{fig:steeringMetrics}. The possible steering effect is not used, because it would be the same for all methods.
      \item the performance is averaged over all groups
      \item the experiment shows that mentioning the attributes in the prompt which is made possible by the interpretable attribute vector presented in this thesis improves the steering performance significantly
      \item it also demonstrates that the newly proposed activation based steering methods (see Section~\ref{sec:approach:steering:activation}) lead to a clear improvement over prompt engineering techniques
    \end{itemize}}%
  \label{table:resultSteeringType}
  \centering
  \resultSteeringType{}%
\end{table}

\begin{itemize}
  \item Table~\ref{table:resultSteeringType} shows the performance of the steering methods averaged over all groups
  \item the \textbf{prompt group steering} as a representation of prompt engineering without the influence of the approach proposed by this method has a clear steering effect that is roughly in the right direction
        \begin{itemize}
          \item the effect is probably not as high as the other methods partly because what the model that generates the group-specific explanation understands of the style of a group might be different from what the median style of the group-specific answers in the training data is
          \item this highlights the importance of using a large amount of divers data to use as the basis for the approach presented in this thesis
        \end{itemize}
  \item \textbf{prompt attribute steering} shows a significantly stronger performance than the prompt group steering, even though the group that is steered towards is not mentioned in the prompt, just the most important knowledge and style attributes
        % TODO: write more? possible problem, because most important in embedding too probably
  \item of all prompt steering methods, \textbf{prompt group attribute steering}, where both the most important attribute and the group itself are part of the system prompt, performs best
        \begin{itemize}
          \item this shows that while the most important attributes lead to a strong steering performance for prompt engineering, mentioning the group increases the steering effect even more
          \item when mentioning the group is not possible however, for example when steering towards the style and knowledge of a single author, the prompt attribute steering will lead to a good performance as well
        \end{itemize}

  \item the activation steering methods use the layers \num{14}, \num{15}, and \num{17} at the same time as well as a \(\lambda\) of \num{0.5} for steering according to the hyperparameter optimization in Section~\ref{sec:evaluation:steering:activationHPO}
  \item even without any prompt engineering, the \textbf{activation base steering} shows a strong steering effect
        \begin{itemize}
          \item the steering direction however is not very good
          \item this could be improved by conducting a more comprehensive hyperparameter optimization
          \item because of the steering direction, the activation base steering is overall the worst performing steering method as it has the largest possible improvement
        \end{itemize}
  \item the \textbf{activation group steering} method shows that by just using a simple prompt engineering approach and mentioning the group in the system prompt in addition the activation steering, the steering performance can be improved significantly
        \begin{itemize}
          \item this increase in performance is largely due to a higher direction correctness, the strength of the steering effect is mostly unchanged
          \item both the direction correctness and especially the steering effect are significantly better than the prompt group steering without the manipulating of the activation space of the model
        \end{itemize}
  \item while activation steering is guiding the \ac{llm} towards the most important attributes for the group, the performance of the \textbf{activation attribute steering} method shows that there is a huge benefit in mentioning the most important attributes in the system prompt additionally
        \begin{itemize}
          \item a large jump from earlier activation steering methods and a huge improvement from the prompt attribute steering method
          \item this could indicate that not the most optimal hyperparameters are chosen for the activation steering and the model therefore needs the engineered system prompt additionally to be guided towards the correct concepts; however, further research is required on this topic
        \end{itemize}
  \item finally, the \textbf{activation group attribute steering} method is the best one
        \begin{itemize}
          \item while mentioning the group does not improve the strength of the steering effect by much, the direction correctness is significantly better compared to the activation attribute steering method
          \item it is clearly better than the prompt group attribute methods, although the difference is not as strong as with the other activation steering methods
        \end{itemize}

  \item overall, the experiment shows that the interpretable attribute vector presented in this thesis can be used to increase the performance of steering through prompt engineering significantly
  \item additionally, the newly proposed activation based steering shows a huge improvement over more conventional prompt engineering techniques
\end{itemize}


\subsection{Comparing the Steering Performance for Different Groups}
\begin{itemize}
  \item while the previous section has evaluated the performance of the different steering methods, this section will focus on the different groups and how well they are suited to being steered towards
  \item Table~\ref{table:resultSteeringGroup} shows the steering performance of the groups averaged over all steering methods
  \item while there consistently is a steering effect for all groups, its strength and the direction correctness varies strongly between the groups
  \item the variation in the strength of the steering effect can be largely explained by the different possible steering effect
        \begin{itemize}
          \item if the style and knowledge of a group is closer to the unsteered generation, the steering effect can not be as large
        \end{itemize}
  \item but even ignoring this, there is a huge difference between groups where the steering works really well like pilots and software engineers and groups where the steering performance is significantly worse like biologists, historians and politicians
        \begin{itemize}
          \item interestingly, these are groups where the knowledge attributes are only of low importance compared to the style attributes as shown in Section~\ref{sec:evaluation:knowledgeAttributes}
          \item there is further research necessary to evaluate whether there is a correlation
        \end{itemize}
\end{itemize}

\begin{table}[ht]
  \caption[]{\begin{itemize}
      \item steering group % TODO:
    \end{itemize}}%
  \label{table:resultSteeringGroup}
  \centering
  \resultSteeringGroup{}%
\end{table}

\begin{itemize}
  \item % TODO:
\end{itemize}

\begin{table}[ht]
  \caption[]{\begin{itemize}
      \item steering group prompt group attribute % TODO:
    \end{itemize}}%
  \label{table:resultSteeringGroupPromptgroupattribute}
  \centering
  \resultSteeringGroupPromptgroupattribute{}%
\end{table}
