\chapter{Conclusion}
\label{sec:conclusion}

\section{Future Work/Improvements} % TODO: better title
\begin{itemize}
  \item use data for the experiments that is guaranteed to be written by specific groups
  \item use more and better data (often a possible improvement in ML/NLP)
  \item probing study to find the best layer for activation steering
  \item further experiments, if a different number of important attributes has an effect on steering (prompt steering and activation steering)
  \item compare \ac{sfam} to human judgements like \citet{patelLearningInterpretableStyle2023}
  \item activation steering not based on the median attribute vector of a group but the attribute vector of a single text to replicate its style.
  \item Multidimensional Analysis (MDA) \newline
        MDA assesses linguistic features across multiple dimensions to profile text styles. For instance, Biber's MDA framework evaluates texts based on factors like formality, narrative style, and informational density. Recent advancements include Neurobiber, a transformer-based system that predicts 96 stylistic features efficiently, facilitating large-scale stylometric research.
\end{itemize}
