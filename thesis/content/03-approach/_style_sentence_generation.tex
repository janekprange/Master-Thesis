\section{Style Sentence Generation}
\label{sec:approach:style_sentence_generation}

The style sentences that form the dimensions of the interpretable style vector are generated by prompting a \acl{llm} in a two-step procedure.

For step one, the \ac{llm} is presented with a zero-shot prompt to produce a description of the text. This step is repeated 92 times, each time with a different prompt that focuses on a specific aspect of the text. 6 prompts are relatively open and only give a loose direction on which style should be described by the model. 84 prompts are targeted towards an explicit stylistic feature. These prompts follow the work of \citet{patelLearningInterpretableStyle2023,tausczikPsychologicalMeaningWords2010} to get the model to focus on a broad variety of style that is important for the study of language.

In addition to these prompts aimed at stylistic features, there are two prompts which are focusing on the knowledge and experience of the author. While the background knowledge of the author is not as important for the group membership detection as the style of the text, it is helpful information for the steering task that is covered in later sections.

After generating the style descriptions, the \ac{llm} is prompted a second time to rewrite the description as a list of sentences. The model is instructed to write each sentence in a consistent form like \enquote{The author is \ldots} or \enquote{The author uses \ldots}.

Furthermore, the model is instructed to avoid negations and examples since these both lead to increasing problems with the clustering process that is described in Section~\ref{sec:approach:clustering}.
Sentences that include examples are potentially problematic because it increases the likelihood of sentences which have the exact same content while being of different shape. % TODO: better wording
Two sentences which differ only in the example are \enquote{The author uses filler words'} and \enquote{The author uses filler words such as 'and', 'or' and 'furthermore'}. While this problem will be reduced by clustering similar sentences, this procedure is not perfect and will be more robust if the model avoids sentences with examples.

Negations on the other hand lead to the problem where the shape of the sentences will be too close while the content is very different. There is a high chance that the sentences \enquote{The author does use long sentences} and \enquote{The author does not use long sentences} will be clustered together because so much of them is the same even though they state the opposite of each other.
Additionally, the dimensions of the final style vector should not include any negated attributes since the expression \enquote{The author uses short sentences} is much clearer than \enquote{The author does not use long sentences}.

Since the model might produce sentences with negations or examples despite the prompt, each of the generated sentences is checked automatically.

Per description, each distinct style sentence is recorded only once, even if the model generates it multiple times. This is done in case the model generates a bad answer where one sentence is repeated many times.

The whole process can be seen in Figure~\ref{fig:style_sentence_generation}.

\begin{figure}[ht]
  \newlength{\mytw}
\settowidth{\mytw}{A few lines of text in a block}
\begin{center}
  \begin{tikzpicture}
    [> = latex', auto,
      block/.style ={
          rectangle,
          draw=black,
          thick,
          % align=flush center,
          rounded corners,
          % minimum height=4em
        },
    ]
    \node[block, text width=0.7\linewidth] (answer) {\textbf{Input Text:}\\If the judge believes the award is too high or too low, the judge can use additur or remittitur to reduce the damages. These are essentially offers to both parties to agree to the change in damages. If both parties do not agree, the court can order a new trial.};

    \node[block, minimum height=3cm, text width=0.4\linewidth, below left=of answer.south] (prompt1) {\textbf{Prompt 1 (Grammar Style):}\\Write a long paragraph describing the unique grammar style of the following passage without referring to specifics about the topic.};
    \node[align=flush center, minimum height=3cm, text width=0.2\linewidth, below=1cm of answer.south] (prompt2) {\ldots};
    \node[block, minimum height=3cm, text width=0.4\linewidth, below right=of answer.south] (prompt3) {\textbf{Prompt 92 (Background Knowledge):}\\Write a description of the background knowledge the author has based on the following passage.};

    \draw[->] (answer.south) -- (prompt1.north);
    \draw[->] (answer.south) -- (prompt2.north);
    \draw[->] (answer.south) -- (prompt3.north);

    \node[block, align=flush center, minimum height=1cm, text width=0.4\linewidth, below=1cm of prompt1] (description1) {Style Description};
    \node[align=flush center, minimum height=1cm, text width=0.2\linewidth, below=1cm of prompt2] (description2) {\ldots};
    \node[block, align=flush center, minimum height=1cm, text width=0.4\linewidth, below=1cm of prompt3] (description3) {Style Description};

    \draw[->] (prompt1.south) -- (description1.north);
    \draw[->] (prompt2.south) -- (description2.north);
    \draw[->] (prompt3.south) -- (description3.north);

    \node[block, text width=0.7\linewidth, below=1cm of description2] (sentence_prompt) {\textbf{Prompt:}\\Rewrite this description as a long list of short sentences describing the author's writing style  where each sentence is in the format of \enquote{The author is X.} or \enquote{The author uses X.}.};

    \draw[->] (description1.south) -- (sentence_prompt.north);
    \draw[->] (description2.south) -- (sentence_prompt.north);
    \draw[->] (description3.south) -- (sentence_prompt.north);

    \node[block, text width=0.95\linewidth, below=1cm of sentence_prompt] (attributes) {\textbf{Style Attributes:}\\
      % The author uses the passive voice. \\
      % The author uses active voice. \\
      % The author uses sentence fragments. \\
      % The author uses run-on sentences. \\
      % The author uses words related to visual perception. \\
      % The author uses words expressing wellness. \\
      % The author uses a neutral tone. \\
      The author uses words related to risk. \\
      The author uses words related to allure. \\
      The author uses words indicating poverty. \\
      The author uses words expressing needs. \\
      The author uses numbers. \\
      The author explains legal concepts. \\
      The author uses words indicating men. \\
      The author uses the word judge. \\
      \ldots
    };

    \draw[->] (sentence_prompt.south) -- (attributes.north);
  \end{tikzpicture}
\end{center}

  % TODO: better caption
  \caption{The process to generate style sentences.}
  \label{fig:style_sentence_generation}
\end{figure}

