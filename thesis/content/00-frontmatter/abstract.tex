% !TeX root = ..\..\thesis.tex
%
\pdfbookmark[0]{Abstract}{Abstract}
\chapter*{Abstract}%
\label{sec:abstract}
\vspace*{-10mm}

% TODO: maybe write better? (and than translate)
\Acp{llm} are used by a large variety of groups of people. To increase the helpfulness of \ac{nlp} methods, it is beneficial to differentiate between these groups and steer generated explanations to an appropriate style and knowledge level to increase their comprehensiveness.

This thesis introduces a novel, unified framework that facilitates interpretable text representation, enabling the detection of group membership and guiding \acp{llm} to generate group-specific explanations within a single, continuous pipeline. The framework extends the interpretable style vector proposed by \citet{patelLearningInterpretableStyle2023} to include knowledge attributes that capture the background knowledge and experience of the author. These combined components form the interpretable attribute vector.

A synthetic dataset with style and knowledge annotations is automatically created to enable the data-driven creation of the interpretable attribute vector. Three models are trained in an automated, unsupervised fashion to generate interpretable attribute vectors and embeddings. The embeddings, created from the attribute vectors, enable meaningful comparisons through Euclidean and cosine distance metrics.

A novel steering method is presented that manipulates the internal activation space of \iac{llm} without requiring any training. This method is compared to prompt engineering-based steering approaches.

Experimental results demonstrate that the proposed approach enables robust group membership detection and identification of important attributes for specific groups. Additionally, the evaluation demonstrates that steering methods benefit from the information provided by the interpretable attribute vector and that the novel, activation-based steering method outperforms prompt engineering techniques.

The source code for this thesis is published on GitLab\footnote{\url{https://gitlab.uni-hannover.de/luhai-nlp/prange25-ma-thesis}}.



% The code is published on GitHub\footnote{\url{https://github.com/janekprange/interpretable-attribute-vector-and-steering}}.


\vspace*{20mm}

\pdfbookmark[0]{Abstract (german)}{Abstract (german)}
\chapter*{Abstract (german)}%
\label{sec:abstract-german}
\vspace*{-10mm}

\begin{otherlanguage}{ngerman}
  \addtolength{\parskip}{-6pt}
  Large Language Models (LLMs) werden von einer Vielzahl unterschiedlicher Personengruppen genutzt. Um die Nützlichkeit von Methodiken im Bereich des Natural Language Processing (NLP) zu erhöhen, ist es vorteilhaft, zwischen diesen Gruppen unterscheiden zu können und die Textgenerierung auf einen geeigneten Stil und ein entsprechendes Wissensniveau für sie auszurichten.

  Diese Arbeit stellt ein neuartiges, einheitliches Framework vor, das eine interpretierbare Textrepräsentation ermöglicht. Dadurch wird sowohl die Erkennung der Gruppenzugehörigkeit als auch die Steuerung von \acp{llm} zur Erzeugung gruppenspezifischer Erklärungen innerhalb einer einzigen, durchgängigen Pipeline ermöglicht. Das Framework erweitert den von \citet{patelLearningInterpretableStyle2023} präsentierten interpretierbaren Stilvektor um Wissensattribute, die das Hintergrundwissen und die Erfahrung der Autorin bzw. des Autors erfassen. Diese kombinierten Komponenten bilden den interpretierbaren Attributvektor.

  Ein synthetischer Datensatz mit Stil- und Wissensannotationen wird automatisch erstellt, um die datengestützte Erstellung des interpretierbaren Attributvektors zu ermöglichen. Drei Modelle werden automatisiert trainiert, um interpretierbare Attributvektoren und -einbettungen zu erzeugen. Die aus den Attributvektoren erstellten Einbettungen ermöglichen sinnvolle Vergleiche anhand von euklidischen und Kosinus-Distanzmetriken.

  Eine neuartige Steuerungsmethode wird vorgestellt, die den internen Aktivierungsraum von einem \ac{llm} manipuliert, ohne dass ein Training erforderlich ist. Diese Methode wird mit auf Prompt Engineering basierenden Steuerungsansätzen verglichen.

  Experimentelle Ergebnisse zeigen, dass die vorgestellte Methode eine robuste Erkennung der Gruppenzugehörigkeit sowie die Identifikation wichtiger Attribute für bestimmte Gruppen ermöglicht. Darüber hinaus zeigt die Evaluation, dass Steuerungsmethoden von den Informationen des interpretierbaren Attributvektors profitieren und dass die neuartige, aktivierungsbasierte Steuerungsmethode Prompt-Engineering-Techniken übertrifft.

  Der Quellcode dieser Arbeit ist auf GitLab veröffentlicht\footnote{\url{https://gitlab.uni-hannover.de/luhai-nlp/prange25-ma-thesis}}.
\end{otherlanguage}
