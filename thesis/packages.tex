% LTEX: enabled=false
\usepackage{subfiles}
\usepackage[utf8]{inputenc}
\usepackage[T1]{fontenc}
\usepackage[ngerman,english]{babel}

\usepackage[document]{ragged2e}
\usepackage[table]{xcolor}
\usepackage{graphicx}
% format numbers
\usepackage{siunitx}
%https://tex.cloud.uni-hannover.de/project/6677e7bc8f12321099572893
\sisetup{
  locale=US,
  group-digits = integer,
  round-mode = places,
  round-precision = 3,
  round-pad = false,
  mode = match, % the numbers are printed in text or math mode matching their surrounding
  detect-weight=true,
  % detect-inline-weight=math,
}
% better itemize and enumerate
\usepackage[inline]{enumitem}
% \setlist[itemize]{nosep, label=-} 
% \setlist[enumerate,1]{label=\alph*)}
\setlist[description]{wide}
% enables \enquote command for better quotation
% \usepackage{csquotes}
\usepackage[printonlyused,nohyperlinks,nolist]{acronym}
\AddToHook{env/quote/begin}{\samepage }

\usepackage{amsmath}
\usepackage{tikz}
\usetikzlibrary{
  angles,
  arrows,
  arrows.meta,
  backgrounds,
  calc,
  fit,
  matrix,
  shadows,
  shapes,
  shapes.geometric,
  shapes.multipart,
  positioning,
}
\usepackage{pgfplots}
\pgfplotsset{compat=1.17}
\usepackage{multicol}

\usepackage[absolute]{textpos}
\setlength{\TPHorizModule}{1mm}
\setlength{\TPVertModule}{1mm}

\usepackage{tabularray}
\UseTblrLibrary{booktabs, siunitx}
% \usepackage{booktabs}
\usepackage[justification=justified, format=plain, singlelinecheck=false]{caption} % increase spacing between table and caption
\usepackage{subcaption}

% All floats are flushed before the next section
% \usepackage[section]{placeins}
% code typesetting
\usepackage{listings}
\usepackage[defaultlines=3,all]{nowidow}


\PassOptionsToPackage{% setup clean thesis style
  backend=biber,              % Biblatex backend
  natbib=true,                % Provide natbib command compatibility
  citestyle=authoryear,       % Citation style within text
  bibstyle=authoryear,        % Citation style in bibliography
  dashed=false,               % Don't replace recurrent author names with a dash
  date=year,                  % Set the publication date in the bibliography to year only
  urldate=ymd,                % Set the date format for the url labels in the bibliography to YYYY-MM-DD
  sorting=nyt,                % Sort citations by name-year-title in the bibliography
  sortcites=false,            % Do not sort the citations within text
  maxcitenames=1,             % Maximum number of names in document-body citations
  maxbibnames=5,              % Maximum number of names in bibliography
  uniquelist=false,           % Don't require unique citation names (year not included) to not overwrite maxcitenames
  defernumbers=true,          % assign numbers to entries when they are printed in the bibliography, not earlier
}{biblatex}
\usepackage{biblatex}
% Don't switch the ordering for the first name in bibliography
\DeclareNameAlias{sortname}{given-family}
\bibliography{literature.bib}

\PassOptionsToPackage{% setup clean thesis style
  figuresep=colon,%
  hangfigurecaption=false,%
  hangsection=true,%
  hangsubsection=true,%
  sansserif=false,%
  configurelistings=true,%
  colorize=full,%
  colortheme=bluemagenta,% upbisg
  configurebiblatex=false,%
  bibfile=literature,%
}{cleanthesis}
\usepackage{cleanthesis}

% required for pgf files generated by matplotlib
\usepackage{import}
\def\mathdefault#1{#1}
\everymath=\expandafter{\the\everymath\displaystyle}

\usepackage{hyperref}
\hypersetup{
  pdftitle={Developing Interpretable Style Vectors to Steer Large Language Models towards Group-Specific Explanation Generation},
  pdfsubject={Master's Thesis},
  pdfauthor={Janek Prange}
  plainpages=false,           %   -
  colorlinks=false,           %   - colorize links?
  pdfborder={0 0 0},          %   -
  breaklinks=true,            %   - allow line break inside links
  bookmarksnumbered=true,     %
  bookmarksopen=true,          %
  pdfpagemode=UseNone,
}

% custom command to produce formatted referenceble research questions
\newcounter{researchQuestion}
\newcounter{researchSubQuestion}[researchQuestion]
\renewcommand{\theresearchQuestion}{\arabic{researchQuestion}}
\renewcommand{\theresearchSubQuestion}{\theresearchQuestion.\arabic{researchSubQuestion}}
\newcommand{\researchQuestion}[2]{%
  \begin{description}
    \refstepcounter{researchQuestion}\label{#1}%
    \item[Research Question~\theresearchQuestion] #2
  \end{description}
}
\newcommand{\researchSubQuestion}[2]{%
  \begin{description}
    \refstepcounter{researchSubQuestion}\label{#1}%
    \item[Research Question~\theresearchSubQuestion] #2
  \end{description}
}